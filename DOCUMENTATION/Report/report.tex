%This is a very basic  BE PROJECT PRELIMINARY Report template.

%############################################# 
%#########Author :  PROJECT###########
%#########COMPUTER ENGINEERING############


\documentclass[oneside,a4paper,12pt]{report}
%\usepackage{showframe}
%\hoffset = 8.9436619718309859154929577464789pt
%\voffset = 13.028169014084507042253521126761pt

\fancypagestyle{plain}{%
  \fancyhf{}
  \fancyfoot[CE]{College_Name, Department of Computer Engineering 2015}
  \fancyfoot[RE]{\thepage}
}
\pagestyle{fancy}
\fancyhead{}
\renewcommand{\headrulewidth}{0pt}
\footskip = 0.625in
\cfoot{}
\rfoot{}

\usepackage[]{hyperref}
\usepackage{tikz}
\usetikzlibrary{arrows,shapes,snakes,automata,backgrounds,petri}

\usepackage{tabularx}

\usepackage[nottoc,notlot,notlof,numbib]{tocbibind}
\usepackage[titletoc]{appendix}
\usepackage{titletoc}
\renewcommand{\appendixname}{Annexure}
\renewcommand{\bibname}{References}

\setcounter{secnumdepth}{5}

\usepackage{float}
\usepackage{subcaption}
\usepackage{multirow}

\usepackage[ruled,vlined]{algorithm2e}

\begin{document}

\setlength{\parindent}{0mm}
\begin{center}
{\bfseries SAVITRIBAI PHULE PUNE UNIVERSITY \\}
 \vspace*{1\baselineskip}
{\bfseries A PROJECT REPORT ON \\}
 \vspace*{2\baselineskip}
{\bfseries \fontsize{16}{12} \selectfont BE PROJECT TITLE \\ \vspace*{2\baselineskip}}
{\fontsize{12}{12} \selectfont SUBMITTED TOWARDS THE
 \\PARTIAL FULFILLMENT OF THE REQUIREMENTS FOR THE AWARD OF THE DEGREE \\

\vspace*{2\baselineskip}}
{\bfseries \fontsize{14}{12} \selectfont BACHELOR OF ENGINEERING (Computer
Engineering) \\
\vspace*{1\baselineskip}} 
{\bfseries \fontsize{14}{12} \selectfont BY \\ 
\vspace*{1\baselineskip}} 
Student Name  \hspace{25 mm} Exam No:  \\
Student Name \hspace{25 mm} Exam No:   \\
Student Name \hspace{25 mm} Exam No:  \\
Student Name \hspace{25 mm} Exam No:\\
\vspace*{2\baselineskip}
{\bfseries \fontsize{14}{12} \selectfont Under The Guidance of \\  
\vspace*{2\baselineskip}} 
Prof. Guide Name\\
\includegraphics[width=100pt]{collegelogo.png} \\
{\bfseries \fontsize{14}{12} \selectfont DEPARTMENT OF COMPUTER ENGINEERING \\
College Name \\
College Address \\
\vspace*{1\baselineskip}
Academic Year
}
\end{center}


\newpage



\begin{figure}[ht]
\centering
\includegraphics[width=100pt]{collegelogo.png}
\end{figure}


{\bfseries \fontsize{14}{12} \selectfont \centerline{College Name}
\centerline{DEPARTMENT OF COMPUTER ENGINEERING}
\vspace*{3\baselineskip}} 


{\bfseries \fontsize{16}{12} \selectfont \centerline{CERTIFICATE} 
\vspace*{3\baselineskip}} 

\centerline{This is to certify that the Project Entitled}
\vspace*{1\baselineskip} 


{\bfseries \fontsize{14}{12} \selectfont \centerline{BE PROJECT TITLE}
\vspace*{1\baselineskip}}

\centerline{Submitted by}
\vspace*{1\baselineskip} 
\centerline{Student Name  \hspace{25 mm} Exam No: } 
\centerline{Student Name \hspace{25 mm} Exam No:  } 
\centerline{Student Name \hspace{25 mm} Exam No: }
\centerline{Student Name \hspace{25 mm} Exam No: }
\vspace*{1\baselineskip} 
is a bonafide work carried out by Students under the supervision of Prof. Guide Name and it
is submitted towards the partial fulfillment of the requirement of Savitribai Phule Pune Univercity,Pune for the award of the degree of Bachelor of Engineering (Computer Engineering). Project.\\\\\\
\bgroup
\def\arraystretch{0.7}
\begin{tabular}{c c c}
External Examiner &  \hspace{10 mm} Prof. Guide Name & \hspace{10 mm} Prof. HOD Name \\					     &  \hspace{10 mm} Internal Guide   &  \hspace{10 mm} H.O.D \\
 &  \hspace{10 mm} Dept. of Computer Engg.  &	\hspace{10 mm}Dept. of Computer Engg.  \\
\end{tabular}
%}
\\\\
Date:\\
Place:


\newpage

%\pictcertificate{TITLE OF BE PROJECT}{Student Name}{Exam Seat No}{Guide Name}
\setcounter{page}{0}
\frontmatter
\cfoot{College Short Form Name, Department of Computer Engineering 2015}
\rfoot{\thepage}
\pagenumbering{Roman}
%\pictack{BE PROJECT TITLE}{Guide Name}

		
{  \newpage {\bfseries \fontsize{14}{12} \selectfont \centerline{Abstract} 
\vspace*{2\baselineskip}} \setlength{\parindent}{11mm} }
{ \setlength{\parindent}{0mm} }
Please Write here One Page Abstract. It should mainly include introduction, motivation,outcome and innovation if any.


{  \newpage {\bfseries \fontsize{14}{12} \selectfont \centerline{Acknowledgments} 
\vspace*{2\baselineskip}} \setlength{\parindent}{11mm} }
{ \setlength{\parindent}{0mm} }
Please Write here Acknowledgment.Example given as\\
\textit{It gives us great pleasure in presenting the preliminary project report 
on {\bfseries \fontsize{12}{12} \selectfont `BE PROJECT TITLE'}.}
\vspace*{1.5\baselineskip}

 \textit{I would like to take this opportunity to thank my internal guide
 \textbf{Prof. Guide Name} for giving me all the help and guidance I needed. I am
 really grateful to them for their kind support. Their valuable suggestions were very helpful.} \vspace*{1.5\baselineskip}

 \textit{I am also grateful to \textbf{Prof. HOD Name}, Head of Computer
 Engineering Department, CollegeName for his indispensable
 support, suggestions.}
\vspace*{1.5\baselineskip}

\textit{In the end our special thanks to \textbf{Other Person Name} for
providing various resources such as  laboratory with all needed software platforms,
continuous Internet connection, for Our Project.}
\vspace*{3\baselineskip} \\
\begin{tabular}{p{8.2cm}c}
&Student Name1\\
&Student Name2\\
&Student Name3\\
&Student Name4\\
&(B.E. Computer Engg.)
%}
\end{tabular}


% \maketitle
\tableofcontents
\listoffigures 
\listoftables



\mainmatter



  \titleformat{\chapter}[display]
{\fontsize{16}{15}\filcenter}
{\vspace*{\fill}
 \bfseries\LARGE\MakeUppercase{\chaptertitlename}~\thechapter}
{1pc}
{\bfseries\LARGE\MakeUppercase}
[\thispagestyle{empty}\vspace*{\fill}\newpage]







\setlength{\parindent}{11mm}
\chapter{Synopsis}

\section{Project Title}
BE Project Title

\section{ Project Option }
Please mention type either industry sponsored, entrepreneur or internal project

\section{Internal Guide}
Prof. Internal Guide Name

\section{ Sponsorship and External Guide} 
Please write if any sponsorship


\section{Technical Keywords (As per ACM Keywords)}
% {\bfseries Technical Key Words:}      
% \begin{itemize}
%   \item 	Cloud Computing
% \item	Service Composition
% \item	Online Web services
% \end{itemize}
Please note ACM Keywords can be found : http://www.acm.org/about/class/ccs98-html \\
Example is given as
\begin{enumerate}
	\item C. Computer Systems Organization 
	\begin{enumerate}
		\item C.2 COMPUTER-COMMUNICATION NETWORKS 
		\begin{enumerate}
			\item C.2.4 Distributed Systems 
			\begin{enumerate}
				\item  Client/server 
\item Distributed applications
\item Distributed databases
\item Network operating systems 
\item Distributed file systems
\item Security and reliability issues in distributed applications
	 		\end{enumerate} 
		\end{enumerate} 
	  

	
	\end{enumerate}
\end{enumerate}



\section{Problem Statement}
\label{sec:problem}
        Define Problem Statement
\section{Abstract}
\begin{itemize}
	\item Abstract (10 to 15 lines)
\end{itemize}

\section{Goals and Objectives}
\begin{itemize}
	\item Objectives
\end{itemize}

	
\section{Relevant mathematics associated with the Project}
\label{sec:math}
System Description:
\begin{itemize} 
\item Input:	 
\item Output:	 
\item Identify data structures, classes, divide and conquer strategies to exploit distributed/parallel/concurrent processing, constraints. 
\item Functions : Identify Objects, Morphisms, Overloading in functions, Functional relations
\item Mathematical formulation if possible
\item Success Conditions:	 
\item Failure Conditions:		
\end{itemize}


\section{Names of Conferences / Journals where papers can be published}
\begin{itemize}
\item  IEEE/ACM Conference/Journal 1 
\item  Conferences/workshops in IITs
\item  Central Universities or SPPU Conferences 
\item IEEE/ACM Conference/Journal 2 
\end{itemize}


\section{Review of Conference/Journal Papers supporting Project idea}
\label{sec:survey}
   Atleast 10 papers + White papers or web references\\
   Brief literature survey [ Description containing important description of at least 10 papers

\section{Plan of Project Execution}
  Using planner or alike project management tool.



\chapter{Technical Keywords}
\section{Area of Project}
Project Area

\section{Technical Keywords}
% {\bfseries Technical Key Words:}      
% \begin{itemize}
%   \item 	Cloud Computing
% \item	Service Composition
% \item	Online Web services
% \end{itemize}
Please note ACM Keywords can be found : http://www.acm.org/about/class/ccs98-html \\
Example is given as

\begin{enumerate}
	\item C. Computer Systems Organization 
	\begin{enumerate}
		\item C.2 COMPUTER-COMMUNICATION NETWORKS 
		\begin{enumerate}
			\item C.2.4 Distributed Systems 
			\begin{enumerate}
				\item  Client/server 
\item Distributed applications
\item Distributed databases
\item Network operating systems 
\item Distributed file systems
\item Security and reliability issues in distributed applications
	 		\end{enumerate} 
		\end{enumerate} 
	  

	
	\end{enumerate}
\end{enumerate}

			
\chapter{Introduction}
\section{Project Idea}
\begin{itemize}
\item Project Idea
\end{itemize}


\section{Motivation of the Project}  
\begin{itemize}
\item Motivation of the Project
\end{itemize}

\section{Literature Survey}
\begin{itemize}
\item Review of the papers, Description , Mathematical Terms 
\end{itemize}


\chapter{Problem Definition and scope}
\section{Problem Statement}
Description of Problem


\subsection{Goals and objectives}  
Goal and Objectives: 
\begin{itemize}
  	\item Overall goals and objectives of software, input and output description with necessary syntax, format etc are described
\end{itemize}

 \subsection{Statement of scope} 
	\begin{itemize}  
	\item	A description of the software with Size of input, bounds on input, input validation, input dependency, i/o state diagram, Major inputs, and outputs are described without regard to implementation detail.
	\item The scope identifies what the product is and is not, what it will and won’t do, what it will and wont contain.
	\end{itemize}

\section{Software context} 
\begin{itemize}
\item The business or product line context or application of the software is to be given
\end{itemize}
\section{Major Constraints}
\begin{itemize}
\item Any constraints that will impact the manner in which the software is to be specified, designed, implemented or tested are noted here.
\end{itemize}

\section{Methodologies of Problem solving and efficiency issues}
\begin{itemize}
	\item The single problem can be solved by different solutions.  This considers the performance parameters for each approach. Thus considers the efficiency issues.
\end{itemize}

\section{Scenario in which multi-core, Embedded and Distributed Computing used}
 Explain the scenario in which multi-core, embedded and distributed computing methodology can be applied.


\section{Outcome}
\begin{itemize}
\item Outcome of the project
\end{itemize}

\section{Applications}
\begin{itemize}
\item Applications of Project
\end{itemize}

\section{Hardware Resources Required}
\begin{table}[!htbp]
\begin{center}
\def\arraystretch{1.5}
  \begin{tabular}{| c | c | c | c |}
\hline
Sr. No. &	Parameter &	Minimum Requirement & Justification \\
\hline
1 &	CPU Speed &	 2 GHz  & Remark Required\\
\hline
2 &	RAM  &	3 GB &  Remark Required\\
 \hline
\end{tabular}
 \caption { Hardware Requirements }
 \label{tab:hreq}
\end{center}

\end{table}


\section{Software Resources Required}
Platform : 
\begin{enumerate}
\item Operating System: 
\item IDE: 
\item Programming Language
\end{enumerate}




\chapter{Project Plan}

\section{Project Estimates}
                 Use Waterfall model and associated streams derived from assignments 1,2, 3, 4 and 5( Annex A and B) for estimation. 
\subsection{Reconciled Estimates}
\subsubsection{Cost Estimate}

\subsubsection{Time Estimates}


\subsection{Project Resources}
          Project resources  [People, Hardware, Software, Tools and other resources] based on Memory Sharing, IPC, and Concurrency derived using appendices to be referred. 

\section{Risk Management w.r.t. NP Hard analysis}
This section discusses Project risks and the approach to managing them.
\subsection{Risk Identification}
For risks identification, review of scope document, requirements specifications and schedule is done. Answers to questionnaire revealed some risks. Each risk is categorized as per the categories mentioned in \cite{bookPressman}. Please refer table \ref{tab:risk} for all the risks. You can refereed following risk identification questionnaire.

\begin{enumerate}
\item Have top software and customer managers formally committed to support the project?
\item Are end-users enthusiastically committed to the project and the system/product to be built?
\item Are requirements fully understood by the software engineering team and its customers?
\item Have customers been involved fully in the definition of requirements?
\item Do end-users have realistic expectations?
\item Does the software engineering team have the right mix of skills?
\item Are project requirements stable?
\item Is the number of people on the project team adequate to do the job?
\item Do all customer/user constituencies agree on the importance of the project and on the requirements for the system/product to be built?
\end{enumerate}

\subsection{Risk Analysis}
The risks for the Project can be analyzed within the constraints of time and quality

\begin{table}[!htbp]
\begin{center}
%\def\arraystretch{1.5}
\def\arraystretch{1.5}
\begin{tabularx}{\textwidth}{| c | X | c | c | c | c |}
\hline
\multirow{2}{*}{ID} & \multirow{2}{*}{Risk Description}	& \multirow{2}{*}{Probability} & \multicolumn{3}{|c|}{Impact} \\ \cline{4-6}
	& & &	Schedule	& Quality	& Overall \\ \hline
1	& Description 1	& Low	& Low	& High	& High \\ \hline
2	& Description 2	& Low	& Low	& High	& High \\ \hline
\end{tabularx}
\end{center}
\caption{Risk Table}
\label{tab:risk}
\end{table}


\begin{table}[!htbp]
\begin{center}
%\def\arraystretch{1.5}
\def\arraystretch{1.5}
\begin{tabular}{| c | c | c |}
\hline
Probability & Value &	Description \\ \hline
High &	Probability of occurrence is &  $ > 75 \% $ \\ \hline
Medium &	Probability of occurrence is  & $26-75 \% $ \\ \hline
Low	& Probability of occurrence is & $ < 25 \% $ \\ \hline
\end{tabular}
\end{center}
\caption{Risk Probability definitions \cite{bookPressman}}
\label{tab:riskdef}
\end{table}

\begin{table}[!htbp]
\begin{center}
%\def\arraystretch{1.5}
\def\arraystretch{1.5}
\begin{tabularx}{\textwidth}{| c | c | X |}
\hline
Impact & Value	& Description \\ \hline
Very high &	$> 10 \%$ & Schedule impact or Unacceptable quality \\ \hline
High &	$5-10 \%$ & Schedule impact or Some parts of the project have low quality \\ \hline
Medium	& $ < 5 \% $ & Schedule impact or Barely noticeable degradation in quality Low	Impact on schedule or Quality can be incorporated \\ \hline
\end{tabularx}
\end{center}
\caption{Risk Impact definitions \cite{bookPressman}}
\label{tab:riskImpactDef}
\end{table}

\subsection{Overview of Risk Mitigation, Monitoring, Management}


Following are the details for each risk.
\begin{table}[!htbp]
\begin{center}
%\def\arraystretch{1.5}
\def\arraystretch{1.5}
\begin{tabularx}{\textwidth}{| l | X |}
\hline 
Risk ID	& 1 \\ \hline
Risk Description	& Description 1 \\ \hline
Category	& Development Environment. \\ \hline
Source	& Software requirement Specification document. \\ \hline
Probability	& Low \\ \hline
Impact	& High \\ \hline
Response	& Mitigate \\ \hline
Strategy	& Strategy \\ \hline
Risk Status	& Occurred \\ \hline
\end{tabularx}
\end{center}
%\caption{Risk Impact definitions \cite{bookPressman}}
\label{tab:risk1}
\end{table}

\begin{table}[!htbp]
\begin{center}
%\def\arraystretch{1.5}
\def\arraystretch{1.5}
\begin{tabularx}{\textwidth}{| l | X |}
\hline 
Risk ID	& 2 \\ \hline
Risk Description	& Description 2 \\ \hline
Category	& Requirements \\ \hline
Source	& Software Design Specification documentation review. \\ \hline
Probability	& Low \\ \hline
Impact	& High \\ \hline
Response	& Mitigate \\ \hline
Strategy	& Better testing will resolve this issue.  \\ \hline
Risk Status	& Identified \\ \hline
\end{tabularx}
\end{center}
\label{tab:risk2}
\end{table}

\begin{table}[!htbp]
\begin{center}
%\def\arraystretch{1.5}
\def\arraystretch{1.5}
\begin{tabularx}{\textwidth}{| l | X |}
\hline 
Risk ID	& 3 \\ \hline
Risk Description	& Description 3 \\ \hline
Category	& Technology \\ \hline
Source	& This was identified during early development and testing. \\ \hline
Probability	& Low \\ \hline
Impact	& Very High \\ \hline
Response	& Accept \\ \hline
Strategy	& Example Running Service Registry behind proxy balancer  \\ \hline
Risk Status	& Identified \\ \hline
\end{tabularx}
\end{center}
\label{tab:risk3}
\end{table}

\section{Project Schedule}  
\subsection{Project task set}  
Major Tasks in the Project stages are:
\begin{itemize}
  \item Task 1:
  \item Task 2: 
  \item Task 3: 
  \item Task 4: 
  \item Task 5: 
\end{itemize}

\subsection{Task network}  
Project tasks and their dependencies are noted in this diagrammatic form.
\subsection{Timeline Chart}  
A project timeline chart is presented. This may include a time line for the entire project.
Above points should be covered  in Project Planner as Annex C and you can mention here Please refer Annex C for the planner

 
\section{Team Organization}
The manner in which staff is organized and the mechanisms for reporting are noted.  
\subsection{Team structure}
The team structure for the project is identified. Roles are defined.

\subsection{Management reporting and communication}
Mechanisms for progress reporting and inter/intra team communication are identified as per assessment sheet and lab time table. 
 
\chapter{Software requirement specification  (SRS is to be prepared using relevant mathematics derived and software engg. Indicators in Annex A and B)}

\section{Introduction}
\subsection{Purpose and Scope of Document}
The purpose of SRS and what it covers is to be stated 

\subsection{Overview of responsibilities of Developer}
What all activities carried out by developer?
  
\section{Usage Scenario}
This section provides various usage scenarios for the system to be developed.  
 \subsection{User profiles}  
The profiles of all user categories are described here.(Actors and their Description)

\subsection{Use-cases}
All use-cases for the software are presented. Description of all main Use cases using use case template is to be provided.

\begin{table}[!htbp]
\begin{center}
%\def\arraystretch{1.5}
\def\arraystretch{1.5}
\begin{tabularx}{\textwidth}{| c | c | X | c | X |}
\hline
Sr No.	& Use Case	& Description	& Actors	& Assumptions \\
\hline
1& Use Case 1 & Description & Actors & Assumption \\
\hline
\end{tabularx}
\end{center}
\caption{Use Cases}
\label{tab:usecase}
\end{table}


\subsection{Use Case View}
Use Case Diagram. Example is given below
\begin{center}
	\begin{figure}[!htbp]
		\centering
		\fbox{\includegraphics[width=\textwidth]{use-case.jpg}}
	  \caption{Use case diagram}
	  \label{fig:usecase}
	\end{figure}
\end{center}  

\section{Data Model and Description}  
\subsection{Data Description}  
Data objects that will be managed/manipulated by the software are described in this section. The database entities or files or data structures  required to be described. For data objects details can be given as below
\subsection{Data objects and Relationships}
  Data objects and their major attributes and relationships among data objects are described using an ERD- like form.

 
 
\section{Functional Model and Description}  
A description of each major software function, along with data flow (structured analysis) or class hierarchy (Analysis Class diagram with class description for object oriented system) is presented. 
\subsection{Data Flow Diagram}  
\subsubsection{Level 0 Data Flow Diagram}
\subsubsection{Level 1 Data Flow Diagram}
 
\subsection{Description of functions}  
A description of each software function is presented. A processing narrative for function n is presented.(Steps)/ Activity Diagrams. For Example Refer \ref{fig:act-dig}



\begin{center}
	\begin{figure}[!htbp]
		\centering
		\fbox{\includegraphics[height=430pt]{activity-dig.jpg}}
	  \caption{Activity diagram}
	  \label{fig:act-dig}
	\end{figure}
\end{center}  



 
\subsection{Activity Diagram:}
\begin{itemize}
	\item	The Activity diagram represents the steps taken.
\end{itemize} 

\subsection{Non Functional Requirements:}
\begin{itemize}
	\item	Interface Requirements
	\item	Performance Requirements
    \item	Software quality attributes such as availability [ related to Reliability], modifiability [includes portability, reusability, scalability] ,  		performance, security, testability and usability[includes self 			adaptability and user adaptability] 
\end{itemize} 

\subsection{State Diagram:}	
  State Transition Diagram\\
Fig.\ref{fig:state-dig} example shows the state transition diagram of Cloud SDK. The states are
represented in ovals and state of system gets changed when certain events occur. The transitions from one state to the other are represented by arrows. The Figure    shows important states and events that occur while creating new project.

\begin{center}
	\begin{figure}[!htbp]
		\centering
		\fbox{\includegraphics[width=230pt]{state-dig.jpg}}
	  \caption{State transition diagram}
	  \label{fig:state-dig}
	\end{figure}
\end{center} 

 \subsection { Restrictions, Limitations, and Constraints:}
 Special issues which impact the specification, design, or implementation of the software are noted here.
 
 \subsection{Software Interface Description}	 
The software interface(s)to the outside world is(are) described.
The requirements for interfaces to other devices/systems/networks/human are stated.



\chapter{Detailed Design Document using Appendix A and B}
 \section{Introduction}  
This document specifies the design that is used to solve the problem of Product.  
\section{Architectural Design}  
	A description of the program architecture is presented. Subsystem design or Block diagram,Package Diagram,Deployment diagram with description is to be presented.

 
  \begin{center}
	\begin{figure}[!htbp]
		\centering
		\fbox{\includegraphics[width=\textwidth]{arch2.jpg}}
	  \caption{Architecture diagram}
	  \label{fig:arch-dig}
	\end{figure}
\end{center} 


\section{Data design (using Appendices A and B)}   
A description of all data structures including internal, global, and temporary data structures, database design (tables), file formats.
\subsection{Internal software data structure}
Data structures that are passed among components the software are described.
\subsection{Global data structure}
Data structured that are available to major portions of the architecture are described.
\subsection{Temporary data structure}
Files created for interim use are described.
\subsection{Database description}
Database(s) / Files created/used  as part of the application is(are) described.


\section{Component Design} 
Class diagrams, Interaction Diagrams, Algorithms. Description of each component description required.

\subsection{Class Diagram}
 \begin{center}
	\begin{figure}[!htbp]
		\centering
		\fbox{\includegraphics[width=450pt]{class-dig.jpg}}
	  \caption{Class Diagram}
	  \label{fig:class-dig}
	\end{figure}
\end{center} 
\subsection{Description of Component 1} 
\subsection{Description of Component 2}
 
\subsection{Software Interface Description}
The software's interface(s) to the outside world are described.
\subsubsection{External machine interfaces}
Interfaces to other machines (computers or devices) are described.
 \subsubsection{External system interfaces}
 Interfaces to other systems, products, or networks are described.
 \subsubsection{Human Interface} 
 An overview of any human interfaces to be designed for the software is presented. 
 \section{User Interface Design} 
 A description of the user interface design of the software is presented.
 \subsection{ Description of the user interface}
 A description of user interface including screen images or prototype is presented.
  \subsection{Interface design rules}
  Conventions and standards used for designing/implementing the user interface are stated.
   \subsection{Components available}
   GUI components available for implementation are noted.
   
\chapter{TEST SPECIFICATION}
Write the testing strategy and procedure for unit, integration, system etc applied for the project only and not the description of testing strategy or method.
Each test must be supported with appropriate test graph, chart
Test graph should be in test specification but test data in Annex G
Refer annex for test data, snapshots

\section{Introduction}  
This section provides an overview of the entire test document. This document describes both the test plan and the test procedure.
\subsection { Goals and Objectives}
Overall goals and objectives of the test process are described.

\subsection { Goals and Objectives}
Overall goals and objectives of the test process are described.
   
\subsection { Statement of scope}
A description of the scope of software testing is developed. Functionality/features/behavior to be tested is noted. In addition any functionality/features/behavior that is not to be tested is also noted.

\subsection {Major constraints}
Any business, product line or technical constraints that will impact the manner in which the software is to be tested are noted here.

\section{Test Plan} 
This section describes the overall testing strategy and the project management issues that are required to properly execute effective tests.
 
\subsection {Softwares( SCIís) to be tested}
The software to be tested is identified by name. Exclusions are noted explicitly. 

\subsection {Testing strategy}
The overall procedure for software testing is described.

\subsection {Unit test cases}
The procedure for unit testing is described for each software component (that will be unit tested) is presented. This section is repeated for all components i.

\subsection {Stubs and/or drivers for component i}
\subsection {Test cases component i}
\subsection {Purpose of tests for component i}
\subsection {Expected results for component i}


\section{Integration testing}
The integration testing procedure is specified.
\subsection {Testing procedure for integration}
\subsection {Stubs and drivers required}
\subsection {Test cases and their purpose}
\subsection {Expected results}

\section{Validation testing}
The validation testing procedure is specified.
\subsection {Testing procedure for validation}
\subsection {Expected results}
\subsection {Pass/fail criterion for all validation tests}

\section{GUI Testing}
The procedure for GUI testing is described for each graphical component is presented. This section is repeated for all components i.
\subsection {Stubs and/or drivers for component i}
\subsection {Test cases component i}
\subsection {Purpose of tests for component i}
\subsection {Expected results for component i}

\section{High-order testing (System Testing)}
The high-order testing procedure is specified. For each of the high order tests specified below, the test procedure, test cases, purpose, specialized requirements and pass/fail criteria are specified. 

It should be noted that not all high-order test methods noted in Sections 8.5.n will be conducted for every project.
\subsection {Recovery testing}
\subsection {Security testing}
\subsection {Stress testing}
\subsection {Performance testing}
\subsection {Alpha/beta testing}
\subsection {Pass/fail criterion for all validation tests}

\section{Test work products}
The work products produced as a consequence of the testing procedure are identified.

\section{Function dependency graph}
Show functional dependencies

\section{Function Scatter diagram}
Plot diagram on x axis function name and on y axis how many times function has been called

\section{Input sequence testing}
Queuing theory used to accept input, scheduling algorithm used to schedule the request and mapping it with function call or ready to execute queue. Test of loosely coupled input requests (multiple requests to process, estimation of scalability, bounds and associated test matrix)
\begin{itemize}
\item	Note : During examination, students may be asked to verify each testing results using breakpoints
\end{itemize}


\chapter{Future Enhancement}
Write future enhancement for the problem statement.
   
 \chapter{Summary and Conclusion}
Write one page summary and conclusion 
\bibliographystyle{ieeetr}
\bibliography{biblo}


\begin{appendices}


% \chapter{ALGORITHMIC DESIGN}
\chapter{Laboratory assignments on Project Analysis of Algorithmic Design}
\begin{itemize}
\item To develop the problem under consideration and justify feasibilty using
concepts of knowledge canvas and IDEA Matrix.\\
Refer \cite{innovationbook} for IDEA Matrix and Knowledge canvas model. Case studies are given in this book. IDEA Matrix is represented in the following form. Knowledge canvas represents about identification  of opportunity for product. Feasibility is represented w.r.t. business perspective.\\ 

\begin{table}[!htbp]
\begin{center}
  \begin{tabular}{| c | c | c | c |}
\hline
 I & D & E & A \\ 
\hline
Increase & Drive & Educate & Accelerate \\
\hline
Improve & Deliver & Evaluate & Associate  \\
 \hline
Ignore & Decrease & Eliminate & Avoid \\
\hline
\end{tabular}
 \caption { IDEA Matrix }
 \label{tab:imatrix}
\end{center}
\end{table}

\item Project problem statement feasibility assessment using NP-Hard, NP-Complete or satisfy ability issues using modern algebra and/or relevant mathematical models.
\item input x,output y, y=f(x)
\end{itemize}










\chapter{Laboratory assignments on Project Quality and Reliability Testing of Project Design}

It should include assignments such as
\begin{itemize}
\item Use of divide and conquer strategies to exploit distributed/parallel/concurrent processing of the above to identify object, morphisms, overloading in functions (if any), and functional relations and any other dependencies (as per requirements).
             It can include Venn diagram, state diagram, function relations, i/o relations; use this to derive objects, morphism, overloading

\item Use of above to draw functional dependency graphs and relevant Software modeling methods, techniques
including UML diagrams or other necessities using appropriate tools.
\item Testing of project problem statement using generated test data (using mathematical models, GUI, Function testing principles, if any) selection and appropriate use of testing tools, testing of UML diagram's reliability. Write also test cases [Black box testing] for each identified functions. 
You can use Mathematica or equivalent open source tool for generating test data. 
\item Additional assignments by the guide. If project type as Entreprenaur, Refer \cite{ehr},\cite{mckinsey},\cite{mckinseyweb}, \cite{govwebsite}
\end{itemize}


\chapter{Project Planner}
\label{app:plan}
Using planner or alike project management tool.




\chapter{Reviewers Comments of Paper Submitted}
(At-least one technical paper must be submitted in Term-I on the project design in the
conferences/workshops in IITs, Central Universities or UoP Conferences or equivalent International Conferences Sponsored by IEEE/ACM)
\begin{enumerate}
\item Paper Title:
\item Name of the Conference/Journal where paper submitted :
\item Paper accepted/rejected : 
\item Review comments by reviewer :
\item Corrective actions if any :  

\end{enumerate}

\chapter{Plagiarism Report}
Plagiarism report


\end{appendices}


\end{document}
